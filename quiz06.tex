\documentclass[a4paper]{article}

% packages
\usepackage[english]{babel}
\usepackage[utf8]{inputenc}
\usepackage{amsmath, amssymb}
\usepackage{color, soul}
\usepackage{hyperref}

% defining pretty green
\definecolor{nicegreen}{RGB}{83, 150, 53}

\title{Quiz 6 \green{Solutions}}

\author{written by Alvin Wan . \href{http://alvinwan.com/cs70}{alvinwan.com/cs70}}

\date{Monday, September 26, 2016}

\begin{document}

% define solution command
\newcommand{\green}[1]{\textcolor{nicegreen}{#1}}
\newcommand{\sol}[1]{{\color{nicegreen} \smallskip\textbf{Solution:} #1}}

% clear solutions
% \renewcommand{\sol}[1]{\smallskip}
% \renewcommand{\green}[1]{\smallskip}

\maketitle

\textbf{This quiz does not count towards your grade.} It exists to simply gauge your understanding. Treat this as though it were a portion of your midterm or final exam.

\section{Fermat's Little Theorem}

\begin{enumerate}
\item Prove that $x^a = x^{a \mod (p-1)} \mod p$.

\sol{
Let $a = m(p-1) + n$, where $n = a (\text{mod} (p-1))$ and $m = \lfloor \frac{a}{p-1} \rfloor$.

Plug in $a$, and we have

\begin{align*}
x^{m(p-1) + n} &= x^{(p-1)m}x^n \mod p\\
&= (x^{p-1})^m x^n
\end{align*}

By Fermat's Little Theorem, $x^{p-1} \equiv 1 \mod p$. Thus,

\begin{align*}
(x^{p-1})^m x^n &= x^n\\
&= x^{a (\text{mod} (p-1))}
\end{align*}
}

\item Solve $2016^{2016^{2016}} \mod 2018$

\sol{
Per the proof in part a (applied recursively), we have 

\begin{align*}
2016^{2016^{2016}} \mod 2018 &= 2016^{2016^{2016 \mod 2016} \mod 2017} \mod 2018\\
&= 2016^{2016^0 \mod 2017} \mod 2018 \\
&= 2016 \mod 2018
\end{align*}
}

\item Let $p$ be prime. Is $a^p \equiv 1 (\text{mod } p) \implies a^{p-1} \equiv 1 (\text{mod } p)$ true?

\sol{\textbf{False}

First, note that if $p$ is prime, then we \textit{always} have the following $a^p \equiv a (\text{mod } p)$. Second, if $a > 0$, $a$ is not divisible by $p$, \textit{and $p$ is still prime}, then we \textit{additionally} have that $a^{p-1} = 1 (\text{mod } p)$.

Although $a^p = a (\text{mod }p)$ is always true when $p$ is prime, $a^{p-1} \equiv 1 (\text{mod } p)$ is not. The latter is true only if we also have that $p$ does not divide $a$, and $a > 0$.
}

\end{enumerate}

\end{document}